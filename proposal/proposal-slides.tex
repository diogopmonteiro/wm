\documentclass{beamer}
\usepackage[utf8]{inputenc}

\begin{document}
	\begin{frame}
		\title{Digital Watermarking for Copyright Owners}
	\author{
	    Diogo Monteiro - 76350 \texttt{diogo.p.monteiro@tecnico.ulisboa.pt} \\
	    João Santos - 76363 \texttt{joao.nuno.santos@tecnico.ulisboa.pt} \\
	    Pedro Reganha - 76489 \texttt{pedro.reganha@tecnico.ulisboa.pt} \\
	}
	\date{Group 4}
		\maketitle
	\end{frame}
  \begin{frame}
    \frametitle{Goals}
    \begin{itemize}
    	\item Tool that allows copyright owners to sign and protect their artwork using digital watermarks.
    	\item Identify the owner of a copyrighted artwork.
    	\item Watermarks resistent to blurring, gaussian noise, lossy compression and other attacks.
    \end{itemize}
  \end{frame}
  \begin{frame}
    \frametitle{Checkpoint and final delivery}
    For the checkpoint, our goal is:

    \begin{itemize}
        \item implement a framework so that adding new algorithms is an easy task;
        \item Cox algorithm;
        \item Discrete Wavelet Transform based algorithm.
    \end{itemize}
    
For the final delivery, we expect:
    \begin{itemize}
        \item everything proposed in the checkpoint;
        \item algorithm capable of image recovery.
    \end{itemize}
  \end{frame}

  \begin{frame}
  \frametitle{Project evaluation}
  	Using a set of images with different resolutions (low, medium, high) the evaluation process is: 
    \begin{enumerate}
        \item apply digital watermarks with every algorithm we have implemented;
        \item simulate the attacks: resampling, JPEG compression, rotation, noise, cropping, low and high pass filters, blur;
        \item extract the watermarks from the tampered artwork;
        \item compare the watermarks;
        \item recover the original image (if the algorithm supports it).
    \end{enumerate}
  \end{frame}
% etc
\end{document}