\documentclass[10pt]{article}

\usepackage[utf8]{inputenc}
\usepackage[hmarginratio=1:1,top=10mm]{geometry} % Document margins
\usepackage{indentfirst}

\begin{document}
\title{Digital Watermarking for Copyright Owners}
\author{
    Diogo Monteiro - 76350 - \texttt{diogo.p.monteiro@tecnico.ulisboa.pt} \\
    João Santos - 76363 - \texttt{joao.nuno.santos@tecnico.ulisboa.pt} \\
    Pedro Reganha - 76489 - \texttt{pedro.reganha@tecnico.ulisboa.pt} \\
}
\date{Group 4}

\maketitle

\section{Objective}
    As digital information may be copied and attacked during storage and transmission, our main goal is to create a tool that allows copyright owners to sign and protect their artwork using digital watermarking methods.
    
   Using these techniques, we should be able to protect against artwork tampering and edition (blurring, noise, lossy compression, etc.) and identify the owner of a copyrighted artwork.
    
    The solution is to implement a set of algorithms that allows to reach the goals defined above.

\section{Checkpoint}
For the checkpoint, our goal is:

    \begin{itemize}
        \item implement a framework so that adding new algorithms is an easy task;
        \item Cox algorithm;
        \item Discrete Wavelet Transform based algorithm.
    \end{itemize}
    
For the final delivery, we expect:
    \begin{itemize}
        \item everything proposed in the checkpoint;
        \item algorithm capable of image recovery.
    \end{itemize}
    
    
\section{Evaluation}
    Using a set of images with different resolutions (low, medium, high) the evaluation process is: 
    \begin{enumerate}
        \item apply digital watermarks with every algorithm we have implemented;
        \item simulate the attacks: resampling, JPEG compression, rotation, noise, cropping, low and high pass filters, blur;
        \item extract the watermarks from the tampered artwork;
        \item compare the watermarks;
        \item recover the original image (if the algorithm supports it).
    \end{enumerate}
    
\section{Future work}
    Create a graphical user interface for the tool, open source the code base and continue to add new algorithms and benchmarks.

\end{document}