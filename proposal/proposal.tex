\documentclass[10pt]{article}

\usepackage[utf8]{inputenc}
\usepackage[hmarginratio=1:1,top=32mm]{geometry} % Document margins


\begin{document}
\title{Forensics Cybersecurity Project Proposal}
\author{
    Diogo Monteiro - 76350 - \texttt{diogo.p.monteiro@tecnico.ulisboa.pt} \\
    João Santos - 76363 - \texttt{js@tecnico.ulisboa.pt} \\
    Pedro Reganha - 76489 - \texttt{pedro.reganha@tecnico.ulisboa.pt} \\
}
\date{Group 4}
\maketitle

\section{Objective}
    Our main goal is to create a tool that allows copywright owners to sign and protect their artwork using digital watermarking methods.
    
    Using these techniques, we should be able to protect against artwork edition (blurring, noise, lossy compression, etc.) and identify the owner of a copyrighted artwork.
    
    The solution is to implement a set of algorithms that allows to reach the goals defined above. Nevertheless, each algorithm has it's own robustness.
\section{Checkpoint}
For the checkpoint, our goal is:

	\begin{itemize}
    	\item Implement a framework so that adding new algorithms is an easy task.
        \item Least-Significant-Bit algorithm.
        \item Cox Algorithm.
        \item the last one.	
    \end{itemize}
    
    For the final delivery, we expect:
    \begin{itemize}
    	\item everything proposed in the checkpoint.
        \item advanced algorithm.
    \end{itemize}
    
\section{Evaluation}

\end{document}